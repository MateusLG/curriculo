\documentclass[letterpaper, 10pt]{article}
\usepackage[utf8]{inputenc}
\usepackage[T1]{fontenc}
\usepackage{titlesec}
\usepackage{marvosym} % Para ícones como telefone e email
\usepackage{hyperref} % Para links clicáveis
\usepackage{ragged2e} % Para controle de alinhamento
\usepackage{geometry}
\usepackage{tabularx}
\usepackage{enumitem} % Para controlar espaçamento de listas

% Define margens
\geometry{
  a4paper,
  margin=1in,
}

% Remove números de página
\pagestyle{empty}

% Define o formato: Bold, Maiúsculas, e uma linha (regra horizontal) abaixo.
\titleformat{\section}[runin]{\bfseries\Large\MakeUppercase}
{\rule{\textwidth}{0.4pt}}{0pt}{}
\titlespacing{\section}{0pt}{12pt}{6pt} % Espaçamento: Esquerda, Antes, Depois
% A macro \titlerule pode ser muito sensível em alguns ambientes; usamos \rule.

\titleformat{\section}{\bfseries\MakeUppercase}
  {}{\rule{\textwidth}{0.4pt}}{} % Linha horizontal como separador
\titlespacing{\section}{0pt}{12pt}{6pt}

% Comando customizado para entradas de Educação e Experiência (texto esquerda, data/local direita)
\newcommand{\resumeEntry}[4]{
  \vspace{4pt}
  \noindent\parbox[t]{0.7\linewidth}{\textbf{#1} \\ \emph{#2}}
  \hfill
  \parbox[t]{0.3\linewidth}{\raggedleft #3 \\ \emph{#4}}
  \vspace{2pt}
}

% Comando customizado para entradas de Projeto (Título • Techs | Data)
\newcommand{\projectEntry}[3]{
  \vspace{4pt}
  \noindent\textbf{#1} \quad \small{• \textit{#2}} \hfill \small{\textbf{#3}}
  \vspace{1pt}
}

\begin{document}

% --- CABEÇALHO ---
\begin{center}
    \LARGE \textbf{Mateus Lira Gomes} \\
    \vspace{3pt}
    \small
    +55 61 98133-2132 \ $|$ \ 
    \href{https://github.com/mateuslg}{github.com/mateuslg} \ $|$ \ 
    \href{https://linkedin.com/in/lgmateus}{linkedin.com/in/lgmateus} \ $|$ \ 
    mateuslira3105@gmail.com
\end{center}

% --------------------------------------------------------------------------------------------------
\vspace{-2pt}
\resumeEntry
    {Universidade de Brasilia - FCTE}
    {Bacharelado em Engenharia de Software}
    {Março. 2024 -- Atualmente}
    {4° Semestre}
    {Brasília, DF.}

\resumeEntry
    {IESB}
    {Bacharelado em Engenharia de Computação}
    {Agosto. 2025 -- Atualmente}
    {3° Semestre}
    {Brasília, DF.}
\vspace{6pt}

% --------------------------------------------------------------------------------------------------
% --------------------------------------------------------------------------------------------------
\section*{EXPERIÊNCIA}
\vspace{-2pt}
\resumeEntry
    {V2Tec Soluções}
    {Desenvolvedor Júnior}
    {Out. 2025 -- Atualmente}
    {}
\vspace{-6pt}
\begin{itemize}[leftmargin=*, nosep]
    \item Codifiquei aplicações, scripts e APIs de alta qualidade, aplicando boas práticas de desenvolvimento e princípios de segurança.
    \item Realizo testes unitários e integrados para garantir a estabilidade e a qualidade contínua do código.
    \item Integro sistemas com bancos de dados relacionais (\textbf{SQL}) e não relacionais (\textbf{NoSQL}), além de APIs externas (\textbf{REST/SOAP}).
    \item Colaboro ativamente com equipes multidisciplinares (design, testes, infra) em um ambiente \textbf{Ágil} (Scrum/Kanban).
    \item Utilizo \textbf{Git} para controle de versão e mantenho a documentação técnica dos processos e do código.
\end{itemize}
\vspace{10pt}

\resumeEntry
    {Academia Unique Athletic Resort}
    {Auxiliar de Contas a Pagar}
    {Jan. 2024 -- Nov. 2024}
    {}
\vspace{-6pt}
\begin{itemize}[leftmargin=*, nosep]
    \item Responsável pelo processamento, lançamento e classificação de notas fiscais, boletos e faturas.
    \item Controlei e agendei pagamentos a fornecedores, garantindo o cumprimento de prazos e evitando multas.
    \item Realização de conciliação bancária diária/semanal, analisando extratos e identificando discrepâncias financeiras.
    \item Organização e manutenção de arquivos digitais e físicos de documentos financeiros para fins de auditoria e controle.
    \item Apoiei na elaboração de relatórios de fluxo de caixa e auxiliei nas rotinas de fechamento financeiro mensal.
\end{itemize}

% --------------------------------------------------------------------------------------------------
% --------------------------------------------------------------------------------------------------
\section*{PROJETOS}
\vspace{-2pt}

% Quantik AI
\noindent\textbf{Quantik AI — API de Assistente Financeiro com IA}
\vspace{5pt}
\begin{itemize}[leftmargin=*, nosep]
    \item Desenvolvido em 2025 para permitir que analistas financeiros obtenham respostas precisas em linguagem natural, baseadas em documentos internos (Notas Fiscais, relatórios e leis).
    \item Construído com uma arquitetura \textbf{RAG (Retrieval-Augmented Generation)} no \textbf{Google Cloud} (Cloud Run e Vertex AI) para otimizar custos e garantir escalabilidade.
\end{itemize}
\vspace{5pt}

% LampIAo
\noindent\textbf{LampIAo — Aplicação Web de Estruturação de Ideias com IA}
\vspace{5pt}
\begin{itemize}[leftmargin=*, nosep]
    \item Aplicação web que usa Inteligência Artificial para iluminar e estruturar ideias: o usuário anota um pensamento e recebe um título e *insights* gerados automaticamente.
    \item Projeto realizado na matéria de Orientação a Objetos na Universidade de Brasília (UnB).
\end{itemize}
\vspace{5pt}

% UnAjuda
\noindent\textbf{UnAjuda — Plataforma Online de Q\&A para Universitários}
\vspace{5pt}
\begin{itemize}[leftmargin=*, nosep]
    \item Plataforma online onde universitários podem fazer perguntas e encontrar respostas sobre diversas matérias ou assuntos de nível superior.
    \item Projeto realizado na matéria de Desenvolvimento de Software na Universidade de Brasília (UnB).
\end{itemize}
\vspace{5pt}

% --------------------------------------------------------------------------------------------------
\section*{HABILIDADES TÉCNICAS E CURSOS}
\vspace{-2pt}
\begin{itemize}[leftmargin=*, nosep, itemsep=2pt]
    \item \textbf{Linguagens:} C, JavaScript, TypeScript, Python
    \item \textbf{Web/Backend:} Flask, FastAPI, NodeJS, Django,
    \item \textbf{Development Tools:} Git, Linux, Docker, Kubernetes, GoogleCloud, Shell 
    \item \textbf{Cursos e Certificações:} \emph{Google Cloud Computing Foundations Certificate}
\end{itemize}
\end{document}
